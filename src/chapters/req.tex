%! suppress = UnresolvedReference


\chapter{\app 的需求分析}\label{ch:req}


\section{应用面向的用户群体}\label{sec:target-user}

本应用的目标用户是佩戴可穿戴动态心电信号监测设备的院外患者。由于心血管疾病的发病率随着年龄增长而增加\cite{Zhongguoxinxieguanjiankangyujibingbaogao20212022},有动态心电监测需求的患者也以中老年人为主。除了常规的功能性需求分析之外,本项目也结合用户群体的特征进行了额外的非功能性需求分析以及相关的设计和实现。


\section{应用的功能性需求分析}\label{sec:func-req}

\todo{应用有哪些功能性需求?(参考开题报告,结合之后实际开发的情况做些调整。)(写完开发那章再回来补应该好写一点。)}


\section{应用的非功能性需求分析}\label{sec:nonfunc-req}

\subsection{应用的易用性需求分析}\label{subsec:usability}

因为大部分中老年的用户没有丰富的移动设备使用经验,所以设计一个简单直观、易于使用的应用程序很重要。界面设计应避免使用复杂的元素,以尽量简洁清晰为目标。

此外,由于一般患者通常不会具备专业的医学知识,所以应用程序内应该尽可能地避免使用过于晦涩难懂的专业术语。同时,对于应用内无法避免使用的部分术语应该提供明显且易于理解的解释,以便用户理解其含义。

\subsection{应用的性能需求分析}\label{subsec:performance}

由于动态心电监测的需要,本应用会长期驻留在系统后台运行。因此,应用程序的性能对于用户体验至关重要。应用程序应该尽可能地保持较低的内存占用和较低的电量消耗,以保证用户的使用体验。

\subsection{应用的兼容性需求分析}\label{subsec:compatibility}

应用程序需要能够在各种主流移动设备上运行。这包括对于Android和iOS两大移动操作系统的支持,以及对于各种型号的移动设备的支持。在应用的设计与实现中,需要充分考虑到不同环境可以为应用提供的可用功能有所差别,以保证应用程序在不同环境下的兼容性。


\section{项目的可行性分析}\label{sec:feasibility}

\todo{项目是否可行?为什么?(事实证明是可行的,但还是得站在项目开始前的视角分析一下。)}
