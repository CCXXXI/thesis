\chapter{\app 的设计}\label{ch:design}


\section{项目的整体架构设计}\label{sec:arch-design}

\subsection{Git子模块简介}\label{subsec:git-submodule}

Git子模块(submodule)允许将一个Git仓库作为另一个Git仓库的子目录。本项目的各个模块有一些共享的部分,使用了子模块来实现这些共享部分的复用与同步。

与Git子模块功能类似的工具还有Git子树(subtree)等,但其他工具对于在主仓库修改之后向子仓库进行推送的这一需求都没有像子模块这样的良好支持,再考虑到整个项目架构的一致性,本项目在内部各个仓库之间的依赖方式上都选择了子模块作为解决方案。

\todo{整体架构设计}


\section{应用的界面与功能设计}\label{sec:app-design}

\todo{功能模块设计}

% 心电图纸包含红色的坐标线,每1mm一小格(用细线分隔),每5mm一大格(用粗线分隔)。横轴一小格表示40ms,一大格表示200ms。纵轴一小格表示0.1mV,一大格表示0.5mV。


\section{应用的数据库设计}\label{sec:db-design}

\todo{数据库设计}
