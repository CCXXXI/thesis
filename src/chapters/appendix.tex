%! suppress = UnresolvedReference


\begin{appendix}

    \begingroup
    \renewcommand{\clearpage}{\relax}
    \listoftodos
    \endgroup

    \listoffigures
    \listoffigureEng


    \chapter*{部分插图的来源}\label{ch:license}

    本文中使用了部分非原创插图,相关信息如下:

    \section*{图~\ref{fig:demos}《\nameref{fig:demos}》}

    图~\ref{fig:demo-0} 取自\cite{chenJiyushenduxuexidexindianfenximoxingdeshejiyuyouhua2021},图~\ref{fig:demo-1} 取自\cite{liuJiyuyidongzhongduanfenxidekechuandairouxingxindianjiancexitong2021},图~\ref{fig:demo-2} 取自\cite{jinPredictingCardiovascularDisease2009},均属于为介绍而适当引用他人已经发表的作品的合理使用情况,本文按排版需要进行了裁剪。

    \section*{图~\ref{fig:ecg-paper}《\nameref{fig:ecg-paper}》}

    原名\textit{Diagram of electrocardiogram paper},由MoodyGroove、Stannered、Markus Kuhn创作,发布为公有领域作品,作品链接为 \url{https://commons.wikimedia.org/wiki/File:ECG_Paper_v2.svg},本文原样使用未作修改。

    \section*{图~\ref{fig:icons}《\nameref{fig:icons}》}

    各图标均取自Google Fonts,链接为 \url{https://fonts.google.com/icons},由Google开源并授权免费使用,本文原样使用未作修改。

    \section*{图~\ref{fig:sinus-rhythm}《\nameref{fig:sinus-rhythm}》}

    原名\textit{ECG of a heart in normal sinus rhythm},由Anthony Atkielski创作,发布为公有领域作品,作品链接为 \url{https://commons.wikimedia.org/wiki/File:SinusRhythmLabels.svg},本文原样使用未作修改。

\end{appendix}
