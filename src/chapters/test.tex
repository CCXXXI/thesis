\chapter{\app 的测试}\label{ch:test}


\section{项目的测试环境}\label{sec:test-env}

\subsection{测试的硬件环境}\label{subsec:test-hardware}

测试共使用了一台Windows电脑、一台macOS电脑、一部Android手机、一个动态心电记录仪。另有部分测试运行于GitHub提供的虚拟环境中,其硬件情况不确定。由于模型可以在CPU上运行,测试时没有显卡方面的需求。Windows电脑是ThinkBook 16 G4+ IAP型号,为主要的开发用设备,也顺便执行了多数测试。macOS电脑为Mac mini M2 Pro型号,用于运行iOS模拟器以进行iOS平台相关测试,;苹果不允许Windows或Linux系统运行其设备模拟器,虽然也有办法绕过限制,但较为不便。Android手机为Redmi Note 8 Pro型号,用于执行蓝牙相关测试,因为Android和iOS模拟器均不支持蓝牙。动态心电记录仪为数创医疗的HA301B型号,用于测试设备连接。GitHub Actions环境用于持续集成中自动执行测试。


\section{测试数据的来源}\label{sec:test-data}

\todo{测试数据的来源}


\section{测试目标与测试方法}\label{sec:test-target}

\todo{测试目标与测试方法}


\section{测试用例}\label{sec:test-case}

\todo{测试用例}


\section{测试结果分析}\label{sec:test-result}

\todo{测试结果分析}
