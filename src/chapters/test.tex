\chapter{\app 的测试}\label{ch:test}


\section{项目的测试环境}\label{sec:test-env}

\subsection{测试的硬件环境}\label{subsec:test-hardware}

测试共使用了一台Windows电脑、一台macOS电脑、一部Android手机、一个动态心电记录仪。另有部分测试运行于GitHub提供的虚拟环境中,其硬件情况不确定。由于模型可以在CPU上运行,测试时没有显卡方面的需求。Windows电脑是ThinkBook 16 G4+ IAP型号,为主要的开发用设备,也顺便执行了多数测试。macOS电脑为Mac mini M2 Pro型号,用于运行iOS模拟器以进行iOS平台相关测试,;苹果不允许Windows或Linux系统运行其设备模拟器,虽然也有办法绕过限制,但较为不便。Android手机为Redmi Note 8 Pro型号,用于执行蓝牙相关测试,因为Android和iOS模拟器均不支持蓝牙。动态心电记录仪为数创医疗的HA301B型号,用于测试设备连接。GitHub Actions环境用于持续集成中自动执行测试。

\subsection{测试的软件环境}\label{subsec:test-software}

本地测试环境中,不需要移动设备的测试都在Windows电脑上进行,其系统版本为Windows 11专业版22H2,测试时使用最新的stable频道的Flutter SDK,最新的Python SDK等;另通过WSL2配置了Linux环境,安装了最新的GCC等工具。需要移动设备的测试多数在模拟器中进行,Android模拟器使用了Pixel 6 Pro API 33型号,iOS模拟器使用了iPhone 14 Pro Max型号;需要真实蓝牙连接的测试在Android手机上执行,为Android 11版本。持续集成环境中各种工具也均使用最新稳定版本。


\section{测试数据的来源}\label{sec:test-data}

测试使用的部分数据是继承至原算法项目的,包括输入数据和模型的预期输出结果。由于其数据量较大,且为异常心电数据,另外从公开数据库中下载了一些正常数据以供测试。具体而言,从PhysioNet下载了CEBSDB数据集\footnote{\url{https://physionet.org/content/cebsdb/1.0.0/}},并提取了其中m001号数据的I、II导联的前10分钟数据,再重采样为125Hz并转换为项目中使用的JSON格式。进行重采样一方面是因为原始数据为5000Hz,文件过大;另一方便是因为HA301B型号心电设备的采样率为125Hz,对数据进行重采样后可以更贴近其情况。


\section{测试用例}\label{sec:test-case}

应用开发过程中,在编写功能代码的同时,也为其编写了许多对应的单元测试用例,本节简单记录这些测试用例的情况。另外,对应用也以演示形式进行了集成测试,但没有明确遵循固定步骤,故不在此列出。

\subsection{Flutter部分}\label{subsec:test-flutter}

Flutter部分的测试基于该框架自带的测试功能。其写法与主流测试库相近,主要差别在于对Flutter UI的测试有特别支持。在测试中,可以使用 @pumpWidget@ 方法只执行某一UI组件相关的代码,方便对各部分分别进行测试。其中的pump一词难以翻译,故以下描述中直接使用原文。

\subsubsection{冒烟测试}

这是最基本的一个测试用例。该用例在开发过程中有少数几次未通过的情况,基本都是类似提交了不完整的代码导致无法通过编译,或者持续集成环境配置出错等在实际运行前就报错的场景。该用例的通过可以确认项目的基本测试环境正常,而其失败则意味着其他用例的执行情况没有详细查看的意义。

\begin{enumerate}
    \item 进行必要的初始化,如数据库初始化等;基本与主函数相同,但数据库文件路径等使用测试用空路径,以模拟应用首次运行的情况。
    \item pump应用主体组件 @AppCore@。
    \item 断言屏幕上可以找到一个 @realTime@ 字符串(在中文环境下为 @实时心电@)。
\end{enumerate}

\subsubsection{标准化距离测试}

这是对~\ref{subsubsec:less-points} 节所述的标准化距离算法的测试。该算法用于计算两点在心电图上的按小格计算的距离,而非屏幕上实际显示的距离。尽管测试私有方法通常不被推荐,但由于通过公开的UI组件测试这种细节较为繁琐,因此本应用的测试中仍然使用了一些 @visibleForTesting@。

测试时定义了一些点作为常量,如表~\ref{tab:spots} 所示。其x坐标含义为以毫秒为单位的时间,y坐标含义为以毫伏为单位的电压。

\begin{table}[!ht]
    \centering
    \bicaption{标准化距离测试中使用的点}{Spots used in normalized distance test}
    \label{tab:spots}
    \begin{tabular}{|c|c|c|}
        \hline
        点 & x  & y   \\
        \hline
        a & 0  & 0   \\
        \hline
        b & 0  & 0.1 \\
        \hline
        c & 40 & 0   \\
        \hline
        d & 40 & 0.1 \\
        \hline
    \end{tabular}
\end{table}

这些点之间的标准化距离的预期结果如表~\ref{tab:expected-distance} 所示。

\begin{table}[!ht]
    \centering
    \bicaption{标准化距离测试的预期结果}{Expected results of normalized distance test}
    \label{tab:expected-distance}
    \begin{tabular}{|c|c|c|}
        \hline
        参数1 & 参数2 & 预期结果         \\
        \hline
        a   & b   & 1            \\
        \hline
        a   & c   & 1            \\
        \hline
        a   & d   & \(\sqrt{2}\) \\
        \hline
    \end{tabular}
\end{table}

\subsubsection{标签间隔测试}

这是对~\ref{subsubsec:label-interval} 节中确定坐标标签文本的间隔的算法的测试。该算法用于确定在心电图上显示坐标标签的间隔,以避免标签过于密集。该算法的输入为心电图的显示时长,输出为标签间隔的时间长度。

由于横屏与竖屏下的测试类似,仅在表~\ref{tab:label-interval} 中列出了竖屏下的测试用例。

\begin{table}[!ht]
    \centering
    \bicaption{标签间隔测试}{Label interval test}
    \label{tab:label-interval}
    \begin{tabular}{|c|c|c|}
        \hline
        显示时长 & 预期标签间隔 \\
        \hline
        1秒   & 1秒     \\
        \hline
        5秒   & 1秒     \\
        \hline
        6秒   & 2秒     \\
        \hline
        10秒  & 2秒     \\
        \hline
    \end{tabular}
\end{table}


\section{测试结果分析}\label{sec:test-result}

\todo{测试结果分析}
