\chapter{\app 的测试}\label{ch:test}


\section{项目的测试环境}\label{sec:test-env}

\subsection{测试的硬件环境}\label{subsec:test-hardware}

测试共使用了一台Windows电脑、一台macOS电脑、一部Android手机、一个动态心电记录仪。另有部分测试运行于GitHub提供的虚拟环境中,其硬件情况不确定。由于模型可以在CPU上运行,测试时没有显卡方面的需求。Windows电脑是ThinkBook 16 G4+ IAP型号,为主要的开发用设备,也顺便执行了多数测试。macOS电脑为Mac mini M2 Pro型号,用于运行iOS模拟器以进行iOS平台相关测试,;苹果不允许Windows或Linux系统运行其设备模拟器,虽然也有办法绕过限制,但较为不便。Android手机为Redmi Note 8 Pro型号,用于执行蓝牙相关测试,因为Android和iOS模拟器均不支持蓝牙。动态心电记录仪为数创医疗的HA301B型号,用于测试设备连接。GitHub Actions环境用于持续集成中自动执行测试。

\subsection{测试的软件环境}\label{subsec:test-software}

本地测试环境中,不需要移动设备的测试都在Windows电脑上进行,其系统版本为Windows 11专业版22H2,测试时使用最新的stable频道的Flutter SDK,最新的Python SDK等;另通过WSL2配置了Linux环境,安装了最新的GCC等工具。需要移动设备的测试多数在模拟器中进行,Android模拟器使用了Pixel 6 Pro API 33型号,iOS模拟器使用了iPhone 14 Pro Max型号;需要真实蓝牙连接的测试在Android手机上执行,为Android 11版本。持续集成环境中各种工具也均使用最新稳定版本。


\section{测试数据的来源}\label{sec:test-data}

测试使用的部分数据是继承至原算法项目的,包括输入数据和模型的预期输出结果。由于其数据量较大,且为异常心电数据,另外从公开数据库中下载了一些正常数据以供测试。具体而言,从PhysioNet下载了CEBSDB数据集\footnote{\url{https://physionet.org/content/cebsdb/1.0.0/}},并提取了其中m001号数据的I、II导联的前10分钟数据,再重采样为125Hz并转换为项目中使用的JSON格式。进行重采样一方面是因为原始数据为5000Hz,文件过大;另一方便是因为HA301B型号心电设备的采样率为125Hz,对数据进行重采样后可以更贴近其情况。


\section{测试用例}\label{sec:test-case}

\todo{测试用例}


\section{测试结果分析}\label{sec:test-result}

\todo{测试结果分析}
