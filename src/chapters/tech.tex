%! suppress = UnresolvedReference


\chapter{相关技术介绍}\label{ch:tech}


\section{Flutter跨平台应用程序开发框架}\label{sec:flutter}

\subsection{使用跨平台应用程序开发框架的意义}\label{subsec:why-framework}

要开发一个\app ,首先要考虑的是该应用支持哪些移动平台。根据2023年3月的数据\cite{MobileOperatingSystem},在中国的移动平台操作系统占有率中,Android系统以76.15\%名列第一,iOS系统以23.28\%位居第二,这两个最主流的移动平台操作系统占据了绝大部分市场份额。因此,本应用开发的目标平台选定为Android和iOS。

Android和iOS开发的各种技术可以分为两大方向:原生应用程序开发和跨平台应用程序开发。构建传统的原生应用程序需要维护两个不同的代码库,分别为Android和iOS平台编写代码,通常意味着需要分别使用Kotlin/Java和Swift/Objective-C来编写两份高度相似的代码。而跨平台应用程序开发框架则可以通过一套代码库同时为Android和iOS平台构建应用程序,降低了开发人员的学习成本,缩减了开发时间,提高了开发效率。

\subsection{Flutter简介}\label{subsec:flutter}

Flutter是一个由Google开源的跨平台应用程序开发框架,仅通过一套代码库就能构建精美的、原生平台编译的跨平台应用\cite{FlutterBuildApps}。

Flutter对Android和iOS平台均有良好支持,此外还支持Windows、Linux、macOS、Web等平台\cite{SupportedDeploymentPlatforms}。基于Flutter框架进行开发不仅可以避免为Android和iOS平台分别编写大量相似的代码,还可以方便未来将部分不依赖于移动端特性的功能(比如历史心电展示等)的相关代码在其他平台(比如Web端)复用。

\subsection{Flutter与其他框架的对比}\label{subsec:flutter-compare}

\todo{为什么选择Flutter而不是其他框架?}

\subsection{Dart语言}\label{subsec:dart}

\todo{什么是Dart?}

\todo{为什么选择Dart,因为这是Flutter所必需的。}

\subsubsection{Dart FFI}\label{subsubsec:ffi}

\todo{什么是Dart FFI?}

\subsection{应用依赖的部分Flutter包}\label{subsec:flutter-packages}

\todo{依赖里有上百个包,介绍哪些?}


\section{项目使用的Python包与C++中的对应替代}\label{sec:python-cpp-packages}

\todo{为什么需要C++中的对应替代?}

\subsection{PyTorch与LibTorch}\label{subsec:pytorch-libtorch}

\todo{PyTorch与LibTorch}

\subsection{NumPy与NumCpp}\label{subsec:numpy-numcpp}

\todo{NumPy与NumCpp}

\subsection{pytest与Catch2}\label{subsec:pytest-catch2}

\todo{pytest与Catch2}

\todo{pytest的官方名称即为全小写形式,即使在位于句首的情况下也是如此。}
