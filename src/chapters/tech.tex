\chapter{相关技术介绍}\label{ch:tech}


\section{Flutter应用程序开发框架}\label{sec:flutter}

\todo{什么是Flutter?}

\subsection{Flutter与其他框架的对比}\label{subsec:flutter-compare}

\todo{为什么选择Flutter而不是其他框架?}

\subsection{Dart语言}\label{subsec:dart}

\todo{什么是Dart?}

\todo{为什么选择Dart,因为这是Flutter所必需的。}

\subsubsection{Dart FFI}\label{subsubsec:ffi}

\todo{什么是Dart FFI?}

\subsection{应用依赖的部分Flutter包}\label{subsec:flutter-packages}

\todo{依赖里有上百个包,介绍哪些?}


\section{项目使用的Python包与C++中的对应替代}\label{sec:python-cpp-packages}

\todo{为什么需要C++中的对应替代?}

\subsection{PyTorch与LibTorch}\label{subsec:pytorch-libtorch}

\todo{PyTorch与LibTorch}

\subsection{NumPy与NumCpp}\label{subsec:numpy-numcpp}

\todo{NumPy与NumCpp}

\subsection{pytest与Catch2}\label{subsec:pytest-catch2}

\todo{pytest与Catch2}

\todo{pytest的官方名称即为全小写形式,即使在位于句首的情况下也是如此。}
