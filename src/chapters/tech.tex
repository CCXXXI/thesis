%! suppress = UnresolvedReference


\chapter{相关技术介绍}\label{ch:tech}


\section{Flutter跨平台应用程序开发框架}\label{sec:flutter}

\subsection{使用跨平台应用程序开发框架的意义}\label{subsec:why-framework}

要开发一个\app ,首先要考虑的是该应用支持哪些移动平台。根据2023年3月的数据\cite{MobileOperatingSystem},在中国的移动平台操作系统占有率中,Android系统以76.15\%名列第一,iOS系统以23.28\%位居第二,这两个最主流的移动平台操作系统占据了绝大部分市场份额。因此,本应用开发的目标平台选定为Android和iOS。

Android和iOS开发的各种技术可以分为两大方向:原生应用程序开发和跨平台应用程序开发。构建传统的原生应用程序需要维护两个不同的代码库,分别为Android和iOS平台编写代码,通常意味着需要分别使用Kotlin/Java和Swift/Objective-C来编写两份高度相似的代码。而跨平台应用程序开发框架则可以通过一套代码库同时为Android和iOS平台构建应用程序,降低了开发人员的学习成本,缩减了开发时间,提高了开发效率。

\subsection{Flutter简介}\label{subsec:flutter}

Flutter是一个由Google开源的跨平台应用程序开发框架,仅通过一套代码库就能构建精美的、原生平台编译的跨平台应用\cite{FlutterBuildApps}。

Flutter对Android和iOS平台均有良好支持,此外还支持Windows、Linux、macOS、Web等平台\cite{SupportedDeploymentPlatforms}。基于Flutter框架进行开发不仅可以避免为Android和iOS平台分别编写大量相似的代码,还可以方便未来将部分不依赖于移动端特性的功能(比如历史心电展示等)的相关代码在其他平台(比如Web端)复用。

\subsection{Flutter与其他框架的对比}\label{subsec:flutter-compare}

Flutter并不是唯一一个支持Android和iOS平台的跨平台应用程序开发框架。在本项目的选型过程中也考虑了其他几个经常被与Flutter进行比较的同类框架,包括React Native\cite{ReactNativeLearn}、Xamarin\cite{XamarinOpensourceMobile}、Ionic Creator\cite{IonicFrameworkCrossPlatform}。

关于这几个框架,最先被对比的是其热度。以近一年的Google趋势作为标准,对比结果如图~\ref{fig:google-trends} 所示。

\begin{figure}[h]
    \centering
    \includegraphics[width=\textwidth]{../assets/google-trends}
    \bicaption{Flutter与其他框架的Google趋势}{Google Trends for Flutter and other frameworks}
    \label{fig:google-trends}
\end{figure}

从图中可以看出,Flutter和React Native的热度远远高于Xamarin和Ionic Creator。热度高意味着其社区更加活跃,生态更为优秀,更容易找到相关的资料和技术支持,也更容易利用社区已经开发过的包,这些对于项目开发来说都是非常重要的。因此,Xamarin和Ionic Creator在本项目的选型过程中最先被排除。

之后需要对比的是Flutter和React Native。从热度上来看,两者的热度都比较平稳,Flutter长期保持在React Native的两倍左右,这意味着Flutter应当被优先考虑,但还不足以作为决定性理由。因此,需要进一步对比这两个框架的特性。

React Native是由Meta(前身为Facebook)于2015年开源的跨平台应用程序开发框架。正如其名称所暗示的,React Native和React(一个流行的Web框架)的关系十分密切,这既是优点也是缺点。从优点的方面来说,React Native非常适合已有React开发经验但没有移动平台开发经验的开发者快速上手,也适合将已有的基于React的Web项目迁移至移动端,但这两点对于本项目来说都没有明显意义。另一方面,由于和React的紧密联系,基于React Native的应用需要使用JavaScript和CSS编写,代码会在JavaScript引擎下解释执行,并通过序列化消息与本机代码桥接通信以渲染原生组件。JavaScript本来就不是一个以性能见长的语言,额外的桥接翻译层更使得其性能较之原生应用程序相去甚远。

作为对比,Flutter是Google于两年后(2017年)开源的,其设计上明显借鉴了React Native,与其一样使用响应式风格的界面编写方式。主要差别在于Flutter是编译成原生代码运行,直接接管并控制屏幕上的每一个像素,由此可以避免使用JavaScript桥接导致的性能问题。

除了上述的性能优势之外,Flutter还在热重载、IDE集成、内置调试工具等方面相较其他框架更为优秀。最终,基于以上所有考虑,本项目选择了Flutter作为开发框架。

\subsection{Dart语言}\label{subsec:dart}

\todo{什么是Dart?}

\todo{为什么选择Dart,因为这是Flutter所必需的。}

\subsubsection{Dart FFI}\label{subsubsec:ffi}

\todo{什么是Dart FFI?}

\subsection{应用依赖的部分Flutter包}\label{subsec:flutter-packages}

\todo{依赖里有上百个包,介绍哪些?}


\section{项目使用的Python包与C++中的对应替代}\label{sec:python-cpp-packages}

\todo{为什么需要C++中的对应替代?}

\subsection{PyTorch与LibTorch}\label{subsec:pytorch-libtorch}

\todo{PyTorch与LibTorch}

\subsection{NumPy与NumCpp}\label{subsec:numpy-numcpp}

\todo{NumPy与NumCpp}

\subsection{pytest与Catch2}\label{subsec:pytest-catch2}

\todo{pytest与Catch2}

\todo{pytest的官方名称即为全小写形式,即使在位于句首的情况下也是如此。}
