\begin{abstract}

    近年来,心血管疾病的患病率持续上升,已成为居民死亡的首位原因。动态心电监测是尽早发现心血管疾病的重要手段。随着小型化的可穿戴式动态心电监测设备的流行和移动设备的普及,移动端心电监测应用的需求愈发明显。另一方面,动态心电图的大量数据对心电信号自动分析算法的需求也更高,加之深度学习模型的快速发展,涌现出了许多用于动态心电分析的智能算法。

    本项目基于已有的动态心电智能分析算法,结合移动应用开发技术,设计并实现了一款面向移动终端的可穿戴动态心电图的智能监测应用。文中首先对项目的背景与意义进行了介绍,然后分析了相关技术并确定了技术选型,接着完成了应用的需求分析,确定了心电图、分析报告、设备管理等需求。之后,按照Flutter特有的Riverpod架构模式划分了应用的各个模块。在开发阶段,基于LibTorch框架对已有的模型进行了移植,然后基于Flutter框架进行了跨平台开发,使应用支持Android和iOS两大主流移动操作系统。最后进行了应用测试,确认应用功能清晰完备,界面简洁美观,易于患者使用。

\end{abstract}
