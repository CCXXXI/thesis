\chapter{总结与展望}\label{ch:conc}


\section{总结}\label{sec:conc}

本文描述了一款面向移动终端的可穿戴动态心电图的智能监测应用。

首先,回顾了国内外心血管疾病现状,介绍了心电监测的用途与原理,以及常规心电图与动态心电图的区别。分析了相关技术与应用的现状,介绍了本项目的主要工作,以及与已有项目的差别。

然后,对项目可能使用的各种技术进行了分析对比,解释了项目的选型理由,并简要介绍了最终使用的各种相关技术。

之后进行了项目的需求分析,明确了目标用户,分析了功能性需求与非功能性需求,并进行了简单的可行性分析。

项目的设计基于Riverpod架构模式,这是一种特定于Flutter框架的架构模式,在各层次模块的划分上与传统模式有一定差异。

在项目的开发阶段,实现了Pan-Tompkins算法,对已有的动态心电图分析算法进行了移植,实现了心电图、心率的显示、分析报告的生成、设备的管理等功能。

最后,对项目的各模块进行了测试,确认了项目的可用性。


\section{展望}\label{sec:pros}

本应用在iOS平台上混合使用了Material风格与Cupertino风格的UI。这是一个受开发时间等因素影响的决策,在大幅降低了工作量的同时,也使得iOS平台上的应用界面相较Android平台上的界面在美观的方面有些欠缺。在未来的开发中,希望能够将更多的界面元素在iOS平台上进行贴近于原生的设计,以提高界面的美观度。

由于缺乏医学专业知识,本应用在分析报告中仅对算法的输出结果进行了简单展示,没有进行更专业的分析。希望未来可以与专业人员进行合作,设计一套对普通患者更友好的分析报告生成方式,以提高应用的实用性。

在心电数据的存储方式上,本应用只是将其当作一般数据直接存入数据库。如果结合心电数据的规律专门设计一套压缩存储方式,应该可以进一步减少存储空间的占用,提高应用的性能。
