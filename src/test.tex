\documentclass[printMode]{ecnuthesis}
% 模版选项:
% printMode     是否开启打印模式, 若缺省则为关闭, 反之则为开启
% 用法示例
% \documentclass[printMode]{ecnuthesis}   (开启打印模式, 适合双面打印)
% \documentclass{ecnuthesis}              (关闭打印模式, 适合提交电子版)

\ecnuSetup {
  % 参数设置
  % 允许采用两种方式设置选项:
  %   1. style/... = ...
  %   2. style = { ... = ... }
  % 注意事项:
  %   1. 请勿在参数设置中出现空行
  %   2. "=" 两侧的空格将被忽略
  %   3. "/" 两侧的空格不会被忽略
  %   4. 请使用英文逗号 "," 分隔选项
  %
  % info 类用于输入论文信息
  info = {
    title = {基于微信小程序的线下课程考勤系统的设计与实现},
    % 中文标题
    %
    titleEN = {Design and Implementation of Offline Course Attendance System Based on WeChat Mini Program},
    % 英文标题
    %
    author = {张三},
    % 作者姓名
    %
    studentID = {10000000000},
    % 作者学号
    %
    department = {理工学院},
    % 学院名称
    %
    major = {计算机科学与技术},
    % 专业名称
    %
    supervisor = {李四},
    % 指导教师姓名
    %
    academicTitle = {教授},
    % 指导教师职称
    %
    year  = 2077,
    % 论文完成年份
    %
    month = 5,
    % 论文完成月份
    %
    keywords = {微信小程序, 课堂考勤, Java 语言, My Sql 数据库},
    % 中文关键词
    % 请使用英文逗号 "," 以分隔
    %
    keywordsEN = {WeChat Mini Program, Class attendance, Java, My Sql},
    % 英文关键词
    % 请使用英文逗号 "," 以分隔
    %
  },
  % style 类用于简单设置论文格式
  style = {
    footnote  = plain,
    % 脚注编号样式
    % 可用选项:
    %   footnote = plain|circled
    % 说明:
    %   plain     脚注的编号仅为数字
    %   circled   脚注的编号为带圆圈数字 (仅限1-10)
    %   (默认选项为 plain )
    %
    numbering = arabic,
    % 章节编号样式
    % 可用选项:
    %   numbering = arabic|alpha|chinese
    % 说明:
    %   arabic    使用数字进行编号 (即理科要求)
    %   alpha     使用字母进行编号 (即外文要求)
    %   chinese   使用汉字进行编号 (即文科要求)
    %   (默认选项为 arabic )
    %
    fontCJK = windows,
    % 中文字体选择
    % 可用选项:
    %   fontCJK = fandol|windows|mac
    % 说明:
    %   fandol    使用 TeX 自带的 fandol 字体
    %   windows   使用 Windows 系统内的字体 (中易)
    %   mac       使用 MacOS 系统内的字体
    %   (默认选项为 fandol )
    %
    fontMath = lm,
    % 数学字体选择
    % 可用选项:
    %   fontMath = lm|times
    % 说明:
    %   lm        使用 TeX 自带的 Latin Modern 数学字体
    %   times     使用 Times 风格的数学字体
    %   (默认选项为 lm )
    %
    bibResource = {./source/thesis-ref.bib},
    % 参考文献数据源
    % 由于使用的是 biber + biblatex , 所以必须明确给出 .bib 后缀名
    %
    logoResource = {./source/inner-cover(contains_font).eps},
    % 封面插图数据源
    % 模版已自带, 位于 ./source/inner-cover(contains_font).eps
    % 默认值为空
  }
}

% 需要的宏包可以自行调用
\usepackage{mwe}

\begin{document}

% 设置前置部分编号
\frontmatter

% 中文摘要环境
\begin{abstract}
  针对传统的课堂点名方式不仅浪费宝贵的时间,而且加重了教师的工作量。因此,借助微信平台为入口,基于微信小程序建立一个课堂考勤系统。文章采用 Java 语言编写,用 Spring Boot 框架实现,选择 My Sql 作为后台数据库,前端用微信开发工具开发,后台使用 Intelli J idea,在降低系统成本的同时并未降低系统的稳定性和可靠性。基于微信小程序开发的课堂考勤系统可以大大节省了课堂和教师的时间成本,并且该系统开发周期短,软件升级维护方便。实测结果表明,本系统运行稳定可靠,达到了快速组建班级、签到、请假等要求。
\end{abstract}

% 英文摘要环境
\begin{abstractEN}
  The traditional classroom roll call not only wastes precious time, but also increases the workload of teachers. Therefore, with the help of the WeChat platform as the entrance, a classroom attendance system based on the WeChat applet is established. The article is written in Java language and implemented with the Spring Boot framework. My Sql is selected as the back-end database, the front-end is developed with WeChat development tools, and the back-end uses Intelli J idea, which reduces the system cost while not reducing the stability and reliability of the system. The classroom attendance system developed based on the WeChat applet can greatly save the time cost of the classroom and teachers, and the system development cycle is short, and the software upgrade and maintenance are convenient. The actual test results show that the system runs stably and reliably, and meets the requirements of rapid class formation, sign-in, and leave.
\end{abstractEN}

% 设置正文编号
\mainmatter

\chapter{绪论}
\section{背景}
\subsection{浅谈中国软件}
\subsubsection{简介}

大字版手机银行、加粗版汇款界面、语音版绑卡流程、刷存折取款ATM机具……随着国务院办公厅《关于切实解决老年人运用智能技术困难的实施方案》的印发,当前金融服务领域已有多项“适老”措施有序推进。

\paragraph{国际性站点支撑}

其中关键问题在于如何把准老年人的切实需求,也就是说,老年人究竟需要什么样的金融服务?只有强化问题导向和需求导向,才能有效解决老年人的难题,保证各项措施落地见效。

\begin{theorem}[素数定理]
  设$x \geqslant 1$, 记$\pi(x)$表示不超过$x$的素数的个数, 则当$x \to \infty$时成立
  \[\pi(x) \sim \frac{x}{\ln x}\]
\end{theorem}

\begin{proof}
  是的,以上很多步骤在当时看起来非常的不严谨,但想不到后人通过严格的数学证明,发现它居然真的是对的。有的时候数学研究,不一定非要局限于你所在的领域,有时也可以先大胆的思考、尝试,然后再回过头来补全它,不要因为害怕而限制住了自己的思维。
  \[\pi(x) \sim \frac{x}{\ln x}\]
  命题得证。
\end{proof}

记者近日走访北京市多家银行网点、大型商超\footnote{商超指超市}、菜市场等老年人高频消费场所后发现,由于老年人所处的年龄段、教育背景、生活习惯不同,其金融服务需求也存在较大差异,因此需要分类施策,采取有针对性、差异性的解决方案。

\begin{lstlisting}[language=C++]
#include <stdio.h>
int main(int argc, char **argv) {
    printf("Hello world!\n");
    return 0;
}
\end{lstlisting}

根据记者的调查结果,老年人的金融服务需求总体可分为三类,分别对应三个不同年龄段。一是刚刚退休的人群,年龄集中在55岁至60岁;二是\footnote{帝鸿指黄帝帝鸿指黄帝帝鸿指黄帝帝鸿指黄帝帝鸿指黄帝帝鸿指黄帝帝鸿指黄帝帝鸿指黄帝帝鸿指黄帝帝鸿指黄帝帝鸿指黄帝帝鸿指黄帝帝鸿指黄帝帝鸿指黄帝帝鸿指黄帝帝鸿指黄帝帝鸿指黄帝帝鸿指黄帝帝鸿指黄帝帝鸿指黄帝}退休10年以内的人群,年龄集中在60岁至70岁;三是退休10年以上的人群,年龄多在70岁以上,有些甚至已进入耄耋之年。\footnote{【正义】:一本云“天下之民,谓之浑沌.”}

“我使用手机银行快10年了,不觉得自己比年轻人用得差。”家住北京市海淀区北太平庄街道的于女士笑称,她今年58岁,已退休3年多,还在岗位时银行卡、手机银行已发展多年。“我40多岁时用银行卡,后来接触手机银行并尝试网购,因为当时年纪相对较轻,学起来容易上手。”

\[
  \lim_{p\to+\infty}\int_{a}^{b}f(x)\sin{px}\,\mathrm{d}x = 0.
\]

尽管手机银行使用熟练,但办理大额资金交易时,这些老年人仍青睐线下渠道。为什么不选择线上渠道?于女士说,一是线上渠道不支持大额转账,二是线上渠道“不安全”,针对老年人的电信诈骗频发。实际上,据记者了解目前各家商业银行均已支持线上大额资金转账,用户拨打银行电话客服即可调整转账限额。

由此可见,除了“信息不对称”因素,制约这类老年人运用智能技术的梗阻还在于外部环境的安全性。因此,解决“不敢用”问题迫在眉睫。国务院此前已召开打击治理电信网络新型违法犯罪工作部际联席会议,部署全国“断卡”行动,严厉打击、整治非法开办贩卖电话卡、银行卡等违法行为,铲除电信网络诈骗案件滋生的土壤。据统计,2020年前10月,全国共破获电信网络诈骗案件15.5万起,同比上升65.6%;拦截处置诈骗电话5100万余次、诈骗短信6.3亿余条,成功止付冻结涉案资金1000余亿元。

这几天心里颇不宁静。今晚在院子里坐着乘凉,忽然想起日日走过的荷塘,在这满月的光里,总该另有一番样子吧。月亮渐渐地升高了,墙外马路上孩子们的欢笑,已经听不见了;妻在屋里拍着闰儿,迷迷糊糊地哼着眠歌。我悄悄地披了大衫,带上门出去。

沿着荷塘,是一条曲折的小煤屑路。这是一条幽僻的路;白天也少人走,夜晚更加寂寞。荷塘四面,长着许多树,蓊蓊郁郁的。路的一旁,是些杨柳,和一些不知道名字的树。没有月光的晚上,这路上阴森森的,有些怕人。今晚却很好,虽然月光也还是淡淡的。

路上只我一个人,背着手踱着。这一片天地好像是我的;我也像超出了平常的自己,到了另一世界里。我爱热闹,也爱冷静;爱群居,也爱独处。像今晚上,一个人在这苍茫的月下,什么都可以想,什么都可以不想,便觉是个自由的人。白天里一定要做的事,一定要说的话,现在都可不理。这是独处的妙处,我且受用这无边的荷香月色好了。

曲曲折折的荷塘上面,弥望的是田田的叶子。\footnote{很牛逼}叶子出水很高,像亭亭的舞女的裙。层层的叶子中间,零星地点缀着些白花,有袅娜地开着的,有羞涩地打着朵儿的;正如一粒粒的明珠,又如碧天里的星星,又如刚出浴的美人。微风过处,送来缕缕清香,仿佛远处高楼上渺茫的歌声似的。这时候叶子与花也有一丝的颤动,像闪电般,霎时传过荷塘的那边去了。叶子本是肩并肩密密地挨着,这便宛然有了一道凝碧的波痕。叶子底下是脉脉的流水,遮住了,不能见一些颜色;而叶子却更见风致了。

月光如流水一般,静静地泻在这一片叶子和花上。薄薄的青雾浮起在荷塘里。叶子和花仿佛在牛乳中洗过一样;又像笼着轻纱的梦。虽然是满月,天上却有一层淡淡的云,所以不能朗照;但我以为这恰是到了好处——酣眠固不可少,小睡也别有风味的。月光是隔了树照过来的,高处丛生的灌木,落下参差的斑驳的黑影,峭楞楞如鬼一般;弯弯的杨柳的稀疏的倩影,却又像是画在荷叶上。塘中的月色并不均匀;但光与影有着和谐的旋律,如梵婀玲上奏着的名曲。

荷塘的四面,远远近近,高高低低都是树,而杨柳最多。这些树将一片荷塘重重围住;只在小路一旁,漏着几段空隙,像是特为月光留下的。树色一例是阴阴的,乍看像一团烟雾;但杨柳的丰姿,便在烟雾里也辨得出。树梢上隐隐约约的是一带远山,只有些大意罢了。树缝里也漏着一两点路灯光,没精打采的,是渴睡人的眼。这时候最热闹的,要数树上的蝉声与水里的蛙声;但热闹是它们的,我什么也没有。

\chapter{忘了}

对于退休10年以内的人群,他们的金融需求有何特点?阻碍他们获取金融服务的“堵点”是什么?记者调查发现,这一群体可细分为两类,一类接受新技术的意愿、能力较强,其金融服务需求与刚退休人群相似;另一类有潜在的智能技术运用需求,如果有人帮助,他们愿意尝试。

在中国工商银行北京广安门支行营业室,上午9点开门后就有老年人陆续前来办理业务。“等待中的老年客户有时会问,为什么排在我后面的那位老人提前走了?他办完业务了吗?这时我们会告诉他,那位老人是通过自助机具办的业务,所以更快,进而询问他是否愿意尝试。”中国工商银行北京广安门支行营业室主任张然说,在银行工作人员的引导与帮助下,不少老年客户对智能服务积极接受,也认可其便捷性与安全性。

但受访老人普遍表示,如果身边没有专业人员指导,自己不会尝试独自使用智能机具,“不会用”“看不清”“怕出错”是主要顾虑。

如何解决以上问题?《实施方案》中明确提出,要推动金融机构、非银行支付机构等优化用户注册、银行卡绑定和支付流程,打造大字版、语音版、简洁版等“适老”手机银行。

“如果手机银行系统识别出客户的年龄在55岁以上,会将常用功能自动切换为老年版。”工商银行相关负责人表示,该行已针对老年群体推出手机银行“幸福生活版”,目前用户数已超1300万。

与标准版相比,老年版手机银行在字号、推荐模块等方面均有不同。记者看到,除页面字号增大外,功能页面中对重点信息的字号作了加大、加粗处理。例如在转账汇款界面,用户输入金额后会弹出对话框,里面显示的是放大后的数字,汇款人姓名也会加粗显示,以便老人识别。

“关注老年客户使用需求是银行应尽的责任,希望能够让老年客户体会到金融智能服务的便捷与安全。”工行上述负责人说,该行老年版手机银行还新增了“安全向导”功能,引导老年客户主动使用“账户安全锁”“转账汇款预警”等智能风险控制服务。

对于退休10年以上甚至进入耄耋之年的老年人,他们的金融服务需求与以上两类人群有何不同?记者调查发现,银行卡、移动互联网技术普及时,上述老人大多已处于退休状态。他们对新技术的认知更为陌生,对传统手段更为依赖,不少人没有银行卡、坚持使用存折,且偏爱现金支付,极少使用移动支付手段。

“每个月15号、16号是发退休金的日子,我早晨会先去菜市场买菜,然后去银行取工资,再回家。”家住北京市西城区广安门外大街的宋先生今年79岁,因为不使用移动支付,拿着存折取现金便成为他“必须要完成的事”。当被问及是否愿意尝试银行卡、移动支付时,老人摆摆手说:“年纪越大越谨慎,存了一辈子钱,我看病、养老都靠它们啦,可不能出意外。”

\begin{figure}
    \centering
    \includegraphics[width=.5\textwidth]{example-image}
    \bicaption{对于退休10年以内的人群,他们的金融需求有何特点?阻碍他们获取金融服务的“堵点”是什么?}{For people who have retired within 10 years, what are the characteristics of their financial needs? What is the "blocking point" that prevents them from accessing financial services?}
  \end{figure}

“在鼓励推广新技术、新方式的同时,要强调保留老年人熟悉的传统服\footnote{摸了摸了摸了}务方式,我想这是一个硬性要求。”国家发展改革委秘书长赵辰昕说,《实施方案》中已明确提出,保留传统金融服务方式,任何单位和个人不得以格式条款、通知、声明、告示等方式拒收现金。

“人民币是我国法定货币,人民币现金是我国境内最基础的支付手段 \cite{Yang_Hy200215} ,任何单位和个人不得拒收。”中国人民银行科技司司长李伟说,央行会将“整治拒收现金工作”作为一项重点工作长期抓下去,强化日常监管,通过暗访巡查等多种方式开展摸底排查,同时建立违法主体名录库,重点跟踪、持续整治。

李伟表示,央行已从现金管理、支付服务、普惠金融三方面入手采取措施,旨在切实提升老年人日常金融服务的可得性和满意度。在普惠金融方面,接下来央行将指导金融机构聚焦老年人日常的高频金融场景,打造线上线下一体化、贴合老年人需求的“适老”金融服务。

% 正文后部分
\backmatter
% 导入参考文献 (需要通过 latexmk 编译后才能显示)
\PrintReference

% 附录环境
\begin{appendix}
  \chapter{杂物}
  感谢天,感谢地,感谢阳光照耀了大地。
\end{appendix}

% 致谢环境
\begin{acknowledgement}
  感谢天,感谢地,感谢阳光照耀了大地。
\end{acknowledgement}

\end{document}
