\documentclass[printMode]{ecnuthesis}
% 模版选项:
% printMode     是否开启打印模式, 若缺省则为关闭, 反之则为开启
% 用法示例
% \documentclass[printMode]{ecnuthesis}   (开启打印模式, 适合双面打印)
% \documentclass{ecnuthesis}              (关闭打印模式, 适合提交电子版)

\ecnuSetup {
  % 参数设置
  % 允许采用两种方式设置选项:
  %   1. style/... = ...
  %   2. style = { ... = ... } 
  % 注意事项: 
  %   1. 请勿在参数设置中出现空行
  %   2. "=" 两侧的空格将被忽略
  %   3. "/" 两侧的空格不会被忽略
  %   4. 请使用英文逗号 "," 分隔选项
  %
  % info 类用于输入论文信息
  info = {
    title = {基于微信小程序的线下课程考勤系统的设计与实现},
    % 中文标题
    %
    titleEN = {Design and Implementation of Offline Course Attendance System Based on WeChat Mini Program},
    % 英文标题
    %
    author = {张三},
    % 作者姓名
    %
    studentID = {10000000000},
    % 作者学号
    %
    department = {理工学院},
    % 学院名称
    %
    major = {计算机科学与技术},
    % 专业名称
    %
    supervisor = {李四},
    % 指导教师姓名
    %
    academicTitle = {教授},
    % 指导教师职称
    %
    year  = 2077,
    % 论文完成年份
    %
    month = 3,
    % 论文完成月份
    %
    graduationYear = 2077,
    % 内封面毕业届别
    % 说明:
    %   若 graduationYear 字段为空,则内封面毕业届别为 year
    %
    keywords = {XML, XSL, 数据批处理, “双赢”},
    % 中文关键词
    % 请使用英文逗号 "," 以分隔
    %
    keywordsEN = {XML, XSL, batch, ‘Win-Win’},
    % 英文关键词
    % 请使用英文逗号 "," 以分隔
    %
  },
  % style 类用于简单设置论文格式
  style = {
    footnote  = plain,
    % 脚注编号样式
    % 可用选项:
    %   footnote = plain|circled
    % 说明:
    %   plain     脚注的编号仅为数字
    %   circled   脚注的编号为带圆圈数字 (仅限1-10)
    %   (默认选项为 plain )
    %
    numbering = arabic,
    % 章节编号样式
    % 可用选项:
    %   numbering = arabic|alpha|chinese
    % 说明:
    %   arabic    使用数字进行编号 (即理科要求)
    %   alpha     使用字母进行编号 (即外文要求)
    %   chinese   使用汉字进行编号 (即文科要求)
    %   (默认选项为 arabic )
    %
    fontCJK = fandol,
    % 中文字体选择
    % 可用选项:
    %   fontCJK = fandol|windows|mac
    % 说明:
    %   fandol    使用 TeX 自带的 fandol 字体
    %   windows   使用 Windows 系统内的字体 (中易)
    %   mac       使用 MacOS 系统内的字体
    %   (默认选项为 fandol )
    %
    fontMath = lm,
    % 数学字体选择
    % 可用选项:
    %   fontMath = lm|times
    % 说明:
    %   lm        使用 TeX 自带的 Latin Modern 数学字体
    %   times     使用 Times 风格的数学字体
    %   (默认选项为 lm )
    %
    % declarationPage = {./source/declaration.pdf},
    % 扫描版声明页 PDF 文件
    % 若该值为空则生成模版预定义的声明页;否则将插入指定路径所对应的 PDF 文件
    % 默认值为空
    %
    bibResource = {./source/thesis-ref.bib},
    % 参考文献数据源
    % 由于使用的是 biber + biblatex , 所以必须明确给出 .bib 后缀名
    %
    logoResource = {./source/inner-cover(contains_font).eps},
    % 封面插图数据源
    % 模版已自带, 位于 ./source/inner-cover(contains_font).eps
    % 默认值为空
  }
}

% 需要的宏包可以自行调用
\usepackage{mwe}
\usepackage{zhlipsum}

\begin{document}

% 设置前置部分编号
\frontmatter

% 中文摘要环境
\begin{abstract}
本文通过一个实际的对日软件外包案件的设计和实现,经历了整个软件开发的过程,包括系统分析、概要设计、详细设计、编码、测试,为某制药企业开发了一个 B2B 的电子商务系统。本系统主要是以该制药企业为购买方,发布企业所需要的货物清单,以本系统为平台,各个供应商进行竞标,由购买方选择购买供应商的货物并下订单发货。由于购买方发布货物需要对大量数据进行操作,因此制作了数据批处理程序,来实现大量数据的导入和导出。系统的在线部分运用了XML和XSL技术,体现了画面实现模版化的优势,使得更有效方便的实现画面重载。该系统的开发过程中,本人主要负责制造在线部分的登陆、Home、估价请求履历检索、取消订单一览、用户信息作成、通知信息作成6个模块以及数据批处理部分的供应商 Master 导入程序。

使用该系统,制药企业可以在众多供应商中选择最价廉物美的原材料,这样大大降低其成本,提高了企业利润。同时,供应商之间也有了相互竞争,可以促进生产,达到“双赢”的效果。
本文最后说明了对日软件开发过程与当今我国软件开发过程的区别,并对我国今后软件事业做了期望和展望。
\end{abstract}

% 英文摘要环境
\begin{abstractEN}
This paper concerns with design and achievement of a software development project for Japanese company, experence the whole process of software development including system analysis, general design, detail design, coding, testing to develop a B2B E-business system for some Medicine Manufacture Enterprise. In this system, the Medicine Manufacture Enterprise as the buyers publish products list.The suppliers use this system as plat to bit products.Then the buyers select suppliers they need and send order to buy products.This system also has some batches to realize huge data importing and exporting.This system's webs use XML and XSL technologies to realize reloading of web templates. During the development of the system, my duty is to develop six webs and one batch.

Using this system, Medicine Manufacture Enterprise can select the best and the cheapest products in several suppliers to descreat cost. At the same time, suppliers will bit each other to inprove production and reach ‘Win-Win’.

At the end of this paper, also explain the deference between Janpanese software development processing and Chinese, and give the deep expectance.
\end{abstractEN}

% 设置正文编号
\mainmatter

\chapter{绪论}

\section{背景}

\subsection{浅谈中国软件}

众所周知,信息产业是……

\subsection{项目背景}

如今,随着互联网的日益流行,……

\subsection{关于 Microsoft Commerce Server 2002}

\subsubsection{简介}

这几天心里颇不宁静。今晚在院子里坐着乘凉,忽然想起日日走过的荷塘,在这满月的光里,总该另有一番样子吧。月亮渐渐地升高了,墙外马路上孩子们的欢笑,已经听不见了;妻在屋里拍着闰儿,迷迷糊糊地哼着眠歌。我悄悄地披了大衫,带上门出去。

\subsubsection{技术特点}

沿着荷塘,是一条曲折的小煤屑路。这是一条幽僻的路;白天也少人走,夜晚更加寂寞。荷塘四面,长着许多树,蓊蓊郁郁的。路的一旁,是些杨柳,和一些不知道名字的树。没有月光的晚上,这路上阴森森的,有些怕人。今晚却很好,虽然月光也还是淡淡的。

\paragraph{国际性站点支撑}

荷塘的四面,远远近近,高高低低都是树,而杨柳最多。这些树将一片荷塘重重围住;只在小路一旁,漏着几段空隙,像是特为月光留下的。树色一例是阴阴的,乍看像一团烟雾;但杨柳的丰姿,便在烟雾里也辨得出。树梢上隐隐约约的是一带远山,只有些大意罢了。树缝里也漏着一两点路灯光,没精打采的,是渴睡人的眼。这时候最热闹的,要数树上的蝉声与水里的蛙声;但热闹是它们的,我什么也没有。

\paragraph{更强大的 B2B}

路上只我一个人,背着手踱着。这一片天地好像是我的;我也像超出了平常的自己,到了另一世界里。我爱热闹,也爱冷静;爱群居,也爱独处。像今晚上,一个人在这苍茫的月下,什么都可以想,什么都可以不想,便觉是个自由的人。白天里一定要做的事,一定要说的话,现在都可不理。这是独处的妙处,我且受用这无边的荷香月色好了。

曲曲折折的荷塘上面,弥望的是田田的叶子。叶子出水很高,像亭亭的舞女的裙。层层的叶子中间,零星地点缀着些白花,有袅娜地开着的,有羞涩地打着朵儿的;正如一粒粒的明珠,又如碧天里的星星,又如刚出浴的美人。微风过处,送来缕缕清香,仿佛远处高楼上渺茫的歌声似的。这时候叶子与花也有一丝的颤动,像闪电般,霎时传过荷塘的那边去了。叶子本是肩并肩密密地挨着,这便宛然有了一道凝碧的波痕。叶子底下是脉脉的流水,遮住了,不能见一些颜色;而叶子却更见风致了。
\begin{lstlisting}[language=C++]
#include <stdio.h>
int main(int argc, char **argv) {
    printf("Hello world!\n");
    return 0;
}
\end{lstlisting}

所谓Commerce Brains,就是B to B专用的、以web为基础的应用程序包。Commerce Brains是由业务组件和Enterprise Manager构成的,如图 \ref{fig-1} 组件分布图所示。
\begin{figure}[htb]
  \centering
  \includegraphics[width=.5\textwidth]{example-image}
  \bicaption{组件分布图}{Component Deploment}\label{fig-1}
\end{figure}

月光如流水一般,静静地泻在这一片叶子和花上。薄薄的青雾浮起在荷塘里。叶子和花仿佛在牛乳中洗过一样;又像笼着轻纱的梦。虽然是满月,天上却有一层淡淡的云,所以不能朗照;但我以为这恰是到了好处——酣眠固不可少,小睡也别有风味的。月光是隔了树照过来的,高处丛生的灌木,落下参差的斑驳的黑影,峭楞楞如鬼一般;弯弯的杨柳的稀疏的倩影,却又像是画在荷叶上。塘中的月色并不均匀;但光与影有着和谐的旋律,如梵婀玲上奏着的名曲。\footnote{这一段将荷塘的精致描写得栩栩如生。}

荷塘的四面,远远近近,高高低低都是树,而杨柳最多。这些树将一片荷塘重重围住;只在小路一旁,漏着几段空隙,像是特为月光留下的。树色一例是阴阴的,乍看像一团烟雾;但杨柳的丰姿,便在烟雾里也辨得出。树梢上隐隐约约的是一带远山,只有些大意罢了。树缝里也漏着一两点路灯光,没精打采的,是渴睡人的眼。这时候最热闹的,要数树上的蝉声与水里的蛙声;但热闹是它们的,我什么也没有。

\subsubsection{产品特点}

忽然想起采莲的事情来了。采莲是江南的旧俗,似乎很早就有,而六朝时为盛;从诗歌里可以约略知道。采莲的是少年的女子,她们是荡着小船,唱着艳歌去的。采莲人不用说很多,还有看采莲的人。那是一个热闹的季节,也是一个风流的季节。梁元帝《采莲赋》里说得好:
\begin{equation}
  \lim_{p\to+\infty}\int_{a}^{b}f(x)\sin{px}\,\mathrm{d}x = 0
\end{equation}

于是妖童媛女,荡舟心许;鷁首徐回,兼传羽杯;欋将移而藻挂,船欲动而萍开。尔其纤腰束素,迁延顾步;夏始春余,叶嫩花初,恐沾裳而浅笑,畏倾船而敛裾。

可见当时嬉游的光景了。这真是有趣的事,可惜我们现在早已无福消受了。

\section{相关工作}

于是又记起《西洲曲》里的句子:

采莲南塘秋,莲花过人头;低头弄莲子,莲子清如水。今晚若有采莲人,这儿的莲花也算得“过人头”了;只不见一些流水的影子,是不行的。这令我到底惦着江南了。——这样想着,猛一抬头,不觉已是自己的门前;轻轻地推门进去,什么声息也没有,妻已睡熟好久了。

\section{论文组织}

月光如流水一般,静静地泻在这一片叶子和花上。薄薄的青雾浮起在荷塘里。叶子和花仿佛在牛乳中洗过一样;又像笼着轻纱的梦。虽然是满月,天上却有一层淡淡的云,所以不能朗照;但我以为这恰是到了好处——酣眠固不可少,小睡也别有风味的。月光是隔了树照过来的,高处丛生的灌木,落下参差的斑驳的黑影,峭楞楞如鬼一般;弯弯的杨柳的稀疏的倩影,却又像是画在荷叶上。塘中的月色并不均匀;但光与影有着和谐的旋律,如梵婀玲上奏着的名曲。


\chapter{系统分析与设计}

\section{需求分析}

这几天心里颇不宁静。今晚在院子里坐着乘凉,忽然想起日日走过的荷塘,在这满月的光里,总该另有一番样子吧。月亮渐渐地升高了,墙外马路上孩子们的欢笑,已经听不见了;妻在屋里拍着闰儿,迷迷糊糊地哼着眠歌。我悄悄地披了大衫,带上门出去。

沿着荷塘,是一条曲折的小煤屑路。这是一条幽僻的路;白天也少人走,夜晚更加寂寞。荷塘四面,长着许多树,蓊蓊郁郁的。路的一旁,是些杨柳,和一些不知道名字的树。没有月光的晚上,这路上阴森森的,有些怕人。今晚却很好,虽然月光也还是淡淡的。

\section{系统概要}

路上只我一个人,背着手踱着。这一片天地好像是我的;我也像超出了平常的自己,到了另一世界里。我爱热闹,也爱冷静;爱群居,也爱独处。像今晚上,一个人在这苍茫的月下,什么都可以想,什么都可以不想,便觉是个自由的人。白天里一定要做的事,一定要说的话,现在都可不理。这是独处的妙处,我且受用这无边的荷香月色好了。

曲曲折折的荷塘上面,弥望的是田田的叶子。叶子出水很高,像亭亭的舞女的裙。层层的叶子中间,零星地点缀着些白花,有袅娜地开着的,有羞涩地打着朵儿的;正如一粒粒的明珠,又如碧天里的星星,又如刚出浴的美人。微风过处,送来缕缕清香,仿佛远处高楼上渺茫的歌声似的。这时候叶子与花也有一丝的颤动,像闪电般,霎时传过荷塘的那边去了。叶子本是肩并肩密密地挨着,这便宛然有了一道凝碧的波痕。叶子底下是脉脉的流水,遮住了,不能见一些颜色;而叶子却更见风致了。

\section{系统流程}

荷塘的四面,远远近近,高高低低都是树,而杨柳最多。这些树将一片荷塘重重围住;只在小路一旁,漏着几段空隙,像是特为月光留下的。树色一例是阴阴的,乍看像一团烟雾;但杨柳的丰姿,便在烟雾里也辨得出。树梢上隐隐约约的是一带远山,只有些大意罢了。树缝里也漏着一两点路灯光,没精打采的,是渴睡人的眼。这时候最热闹的,要数树上的蝉声与水里的蛙声;但热闹是它们的,我什么也没有。

\chapter{概要设计}

\section{业务流程}

那年冬天,祖母死了,父亲的差使也交卸了,正是祸不单行的日子。我从北京到徐州,打算跟着父亲奔丧回家。到徐州见着父亲,看见满院狼藉的东西,又想起祖母,不禁簌簌地流下眼泪。父亲说:“事已如此,不必难过,好在天无绝人之路!”

回家变卖典质,父亲还了亏空;又借钱办了丧事。这些日子,家中光景很是惨淡,一半为了丧事,一半为了父亲赋闲。丧事完毕,父亲要到南京谋事,我也要回北京念书,我们便同行。

\section{功能模块介绍}

如图 \ref{fig-2} 系统用例图所示,系统的业务流程分为两部分,购买WF和In购买。
\begin{figure}[htb]
  \centering
  \includegraphics[width=.5\textwidth]{example-image}
  \bicaption{系统用例图}{System Use Case}\label{fig-2}
\end{figure}

曲曲折折的荷塘上面,弥望的是田田的叶子。叶子出水很高,像亭亭的舞女的裙。层层的叶子中间,零星地点缀着些白花,有袅娜地开着的,有羞涩地打着朵儿的;正如一粒粒的明珠,又如碧天里的星星,又如刚出浴的美人。微风过处,送来缕缕清香,仿佛远处高楼上渺茫的歌声似的。这时候叶子与花也有一丝的颤动,像闪电般,霎时传过荷塘的那边去了。叶子本是肩并肩密密地挨着,这便宛然有了一道凝碧的波痕。叶子底下是脉脉的流水,遮住了,不能见一些颜色;而叶子却更见风致了。

\chapter{详细设计}

\section{数据库的设计}

到南京时,有友人约去游逛,勾留了一日;第二日上午便须渡江到浦口,下午上车北去。父亲因为事忙,本已说定不送我,叫旅馆里一个熟识的茶房陪我同去。他再三嘱咐茶房,甚是仔细。但他终于不放心,怕茶房不妥帖;颇踌躇了一会。其实我那年已二十岁,北京已来往过两三次,是没有什么要紧的了。他踌躇了一会,终于决定还是自己送我去。我两三劝他不必去;他只说,“不要紧,他们去不好!”

\section{详细设计书}

我们过了江,进了车站。我买票,他忙着照看行李。行李太多了,得向脚夫行些小费才可过去。他便又忙着和他们讲价钱。我那时真是聪明过分,总觉他说话不大漂亮,非自己插嘴不可,但他终于讲定了价钱;就送我上车。他给我拣定了靠车门的一张椅子;我将他给我做的紫毛大衣铺好坐位。他嘱我路上小心,夜里警醒些,不要受凉。又嘱托茶房好好照应我。我心里暗笑他的迂;他们只认得钱,托他们只是白托!而且我这样大年纪的人,难道还不能料理自己么?唉,我现在想想,那时真是太聪明了!
\begin{equation}
  \sin^2 x + \cos^2 x = 1
\end{equation}

我说道,“爸爸,你走吧。”他望车外看了看说:“我买几个橘子去。你就在此地,不要走动。”我看那边月台的栅栏外有几个卖东西的等着顾客。

\begin{theorem}[素数定理]
  设$x \geqslant 1$, 记$\pi(x)$表示不超过$x$的素数的个数, 则当$x \to \infty$时成立
  \[\pi(x) \sim \frac{x}{\ln x}\]
\end{theorem}

\begin{proof}
  是的,以上很多步骤在当时看起来非常的不严谨,但想不到后人通过严格的数学证明,发现它居然真的是对的。有的时候数学研究,不一定非要局限于你所在的领域,有时也可以先大胆的思考、尝试,然后再回过头来补全它,不要因为害怕而限制住了自己的思维。
  \[\pi(x) \sim \frac{x}{\ln x}\]
  命题得证。
\end{proof}

走到那边月台,须穿过铁轨,须跳下去又爬上去。父亲是一个胖子,走过去自然要费事些。我本来要去的,他不肯,只好让他去。我看见他戴着黑布小帽,穿着黑布大马褂,深青布棉袍,蹒跚地走到铁轨边,慢慢探身下去,尚不大难。可是他穿过铁轨,要爬上那边月台,就不容易了。他用两手攀着上面,两脚再向上缩;他肥胖的身子向左微倾,显出努力的样子。这时我看见他的背影,我的泪很快地流下来了。我赶紧拭干了泪。怕他看见,也怕别人看见。我再向外看时,他已抱了朱红的橘子往回走了。过铁轨时,他先将橘子散放在地上,自己慢慢爬下,再抱起橘子走。

\section{界面设计}

这几天心里颇不宁静。今晚在院子里坐着乘凉,忽然想起日日走过的荷塘,在这满月的光里,总该另有一番样子吧。月亮渐渐地升高了,墙外马路上孩子们的欢笑,已经听不见了;妻在屋里拍着闰儿,迷迷糊糊地哼着眠歌。我悄悄地披了大衫,带上门出去。

沿着荷塘,是一条曲折的小煤屑路。这是一条幽僻的路;白天也少人走,夜晚更加寂寞。荷塘四面,长着许多树,蓊蓊郁郁的。路的一旁,是些杨柳,和一些不知道名字的树。没有月光的晚上,这路上阴森森的,有些怕人。今晚却很好,虽然月光也还是淡淡的。

\chapter{系统实现}

\section{界面}

到这边时,我赶紧去搀他。他和我走到车上,将橘子一股脑儿放在我的皮大衣上。于是扑扑衣上的泥土,心里很轻松似的。过一会说:“我走了,到那边来信!”我望着他走出去。

他走了几步,回过头看见我,说:“进去吧,里边没人。”等他的背影混入来来往往的人里,再找不着了,我便进来坐下,我的眼泪又来了。


\chapter{关键算法与数据结构}

近几年来,父亲和我都是东奔西走,家中光景是一日不如一日。他少年出外谋生,独力支持,做了许多大事。哪知老境却如此颓唐!他触目伤怀,自然情不能自已。情郁于中,自然要发之于外;家庭琐屑便往往触他之怒。他待我渐渐不同往日。

但最近两年的不见,他终于忘却我的不好,只是惦记着我,惦记着我的儿子。我北来后,他写了一信给我,信中说道:“我身体平安,惟膀子疼痛厉害,举箸提笔,诸多不便,大约大去之期不远矣。”我读到此处,在晶莹的泪光中,又看见那肥胖的、青布棉袍黑布马褂的背影。唉!我不知何时再能与他相见!

\chapter{系统测试}

\section{测试计划}

\subsection{单元测试}

单元测试主要是对画面显示的测试,主要测试点如表 \ref{table-1} 所示。

\begin{table}[htb]
\centering
\bicaption{单体测试表}{Unit Test Case}\label{table-1}
\begin{tabular}{|l|l|}
  \hline
  输入系 & \begin{tabular}[c]{@{}l@{}}输入正常值\\ (最小值\textless{}值\textless{}最大值)\end{tabular} \\ \hline
      & \begin{tabular}[c]{@{}l@{}}输入系统异常值\\ (值=最小值-1,值=最大值+1)\end{tabular}  \\ \hline
      & \begin{tabular}[c]{@{}l@{}}输入系统正常值(边际测试)\\ (值=最小值,值=最大值)\end{tabular} \\ \hline
      & 未输入值 \\ \hline
  显示系 & 标题(Title)显示                                                                     \\ \hline
      & 说明文显示                                                                           \\ \hline
      & 显示相应业务的左菜单  \\ \hline
      & 显示Text项目(不换行) \\ \hline
      & 按需要显示排序按钮 \\ \hline
      & 添加附件参照  \\ \hline
  操作系 & IE返回按钮按下后画面迁移                                                                   \\ \hline
      & IE $\times$按钮按下后,系统终了                                                           \\ \hline
      & 下一画面迁移按钮按下后的迁移                                                                  \\ \hline
      & 上一画面迁移按钮按下后的迁移                                                                  \\ \hline
\end{tabular}
\end{table}

\subsection{结合测试}
\subsection{模拟测试}
\subsection{随机测试}


\chapter{测试用例与测试报告}

\chapter{总结与展望}


% 正文后部分
\backmatter
% 导入参考文献 (需要通过 latexmk 编译后才能显示)
\PrintReference

% 附录环境
\begin{appendix}
  \begin{enumerate}
    \item 部分函数代码
    \begin{enumerate}[label=(\arabic*)]
      \item LoadMessageFile函数(过程),用于读取信息XML文件,取得相应节点的内容,并通过引用参数返回值。
    \end{enumerate}
  \end{enumerate}
\end{appendix}

% 致谢环境
\begin{acknowledgement}
  感谢天,感谢地,感谢阳光照耀了大地。
\end{acknowledgement}

\end{document}